\hypertarget{a00056}{\subsection{6\-Lo\-W\-P\-A\-N implementation}
\label{a00056}\index{6\-Lo\-W\-P\-A\-N implementation@{6\-Lo\-W\-P\-A\-N implementation}}
}


6lowpan is a \-Working \-Group in \-I\-E\-T\-F which defines the use of \-I\-Pv6 on \-I\-E\-E\-E 802.\-15.\-4 links.  


6lowpan is a \-Working \-Group in \-I\-E\-T\-F which defines the use of \-I\-Pv6 on \-I\-E\-E\-E 802.\-15.\-4 links. \-Our implementation is based on \-R\-F\-C4944 {\itshape \-Transmission of \-I\-Pv6 \-Packets over \-I\-E\-E\-E 802.\-15.\-4 \-Networks\/}, draft-\/hui-\/6lowpan-\/interop-\/00 {\itshape \-Interoperability \-Test for 6\-Lo\-W\-P\-A\-N\/}, and draft-\/hui-\/6lowpan-\/hc-\/01 {\itshape \-Compression format for \-I\-Pv6 datagrams in 6lowpan \-Networks\/}.



\hypertarget{a00056_drafts}{}\subsubsection{\-Specifications implemented}\label{a00056_drafts}
\begin{DoxyNote}{\-Note}
\-We currently only support 802.\-15.\-4 64-\/bit addresses.
\end{DoxyNote}
\hypertarget{a00056_rfc4944}{}\paragraph{\-R\-F\-C 4944}\label{a00056_rfc4944}
\-R\-F\-C4944 defines address configuration mechanisms based on 802.\-15.\-4 16-\/bit and 64-\/bit addresses, fragmentation of \-I\-Pv6 packets below \-I\-P layer, \-I\-Pv6 and \-U\-D\-P header compression, a mesh header to enable link-\/layer forwarding in a mesh under topology, and a broadcast header to enable broadcast in a mesh under topology.

\-We implement addressing, fragmentation, and header compression. \-We support the header compression scenarios defined in draft-\/hui-\/6lowpan-\/interop-\/00. \-This draft defines an interoperability scenario which was used between \-Arch\-Rock and \-Sensinode implementations.

\-We do not implement mesh under related features, as we target route over techniques.\hypertarget{a00056_hc01}{}\paragraph{draft-\/hui-\/6lowpan-\/hc-\/01}\label{a00056_hc01}
draft-\/hui-\/6lowpan-\/hc-\/01 defines a stateful header compression mechanism which should soon deprecate the stateless header compression mechanism defined in \-R\-F\-C4944. \-It is much more powerfull and flexible, in particular it allows compression of some multicast addresses and of all global unicast addresses.



\hypertarget{a00056_general}{}\subsubsection{\-Implementation overview}\label{a00056_general}
6lowpan does not run as a separate process. \-It is called by the \-M\-A\-C process when a 6lowpan packet is received, and by the tcpip process when an \-I\-Pv6 packet needs to be sent.

\-It is initialized from the \-M\-A\-C process, which calls sicslowpan\-\_\-init (giving as argument a pointer to the mac\-\_\-driver structure).

\-The main 6lowpan functions are implemented in the sicslowpan.\-h and sicslowpan.\-c files. \-They are used to format packets between the 802.\-15.\-4 and the \-I\-Pv6 layers.

6lowpan also creates a few \-I\-Pv6 and link-\/layer dependencies which are detailed in the next section.



\hypertarget{a00056_implementation}{}\subsubsection{\-Implementation details}\label{a00056_implementation}
\hypertarget{a00056_Addressing}{}\paragraph{\-Addressing}\label{a00056_Addressing}
{\bfseries \-Link-\/layer addresses}\par
 \-The format of a 802.\-15.\-4 address is defined in uip.\-h. 
\begin{DoxyCode}
/** \brief 64 bit 802.15.4 address */
struct uip_802154_shortaddr {
  uint8_t addr[2];
};
/** \brief 16 bit 802.15.4 address */
struct uip_802154_longaddr {
  uint8_t addr[8];
};
/** \brief 802.15.4 address */
typedef struct uip_802154_longaddr uip_lladdr_t;
#define UIP_802154_SHORTADDR_LEN 2
#define UIP_802154_LONGADDR_LEN  8
#define UIP_LLADDR_LEN UIP_802154_LONGADDR_LEN
\end{DoxyCode}


{\bfseries \-Neighbor \-Discovery \-Link \-Layer \-Address options }\par
 \-The format of \-N\-D link-\/layer address options depends on the length of the link-\/layer addresses. 802.\-15.\-4 specificities regarding link-\/layer address options are implemented in uip-\/nd6.\-h. 
\begin{DoxyCode}
#define UIP_ND6_OPT_SHORT_LLAO_LEN     8
#define UIP_ND6_OPT_LONG_LLAO_LEN      16
#define UIP_ND6_OPT_LLAO_LEN UIP_ND6_OPT_LONG_LLAO_LEN 
\end{DoxyCode}


{\bfseries \-Address \-Autoconfiguration}\par
 \-The address autoconfiguration mechanism also depends on the format of the link-\/layer address. \-The dependency is reflected in the \#uip\-\_\-netif\-\_\-addr\-\_\-autoconf\-\_\-set function in \#uip-\/netif.\-c. 
\begin{DoxyCode}
#if (UIP_LLADDR_LEN == 8)
  memcpy(ipaddr->u8 + 8, lladdr, UIP_LLADDR_LEN);
  ipaddr->u8[8] ^= 0x02;  
\end{DoxyCode}
\hypertarget{a00056_io}{}\paragraph{\-Packet Input/\-Output}\label{a00056_io}
\-At initialization, the \#input function in sicslowpan.\-c is set as the function to be called by the \-M\-A\-C upon packet reception. \-The \#output function is set as the tcpip\-\_\-output function.\par
 \-At packet reception, the link-\/layer copies the 802.\-15.\-4 payload in the rime buffer, and sets its length. \-It also stores the source and destination link-\/layer addresses as two rime addresses. 
\begin{DoxyCode}
packetbuf_copyfrom(&rx_frame.payload, rx_frame.payload_length);
packetbuf_set_datalen(rx_frame.payload_length);
packetbuf_set_addr(PACKETBUF_ADDR_RECEIVER, (const rimeaddr_t *)&rx_frame.
      dest_addr); 
packetbuf_set_addr(PACKETBUF_ADDR_SENDER, (const rimeaddr_t *)&rx_frame.
      src_addr);
\end{DoxyCode}
 \-It then calls the sicslowpan \#input function. \-Similarly, when the \-I\-Pv6 layer has a packet to send over the radio, it puts the packet in uip\-\_\-buf, sets uip\-\_\-len and calls the sicslowpan \#output function.\hypertarget{a00056_frag}{}\paragraph{\-Fragmentation}\label{a00056_frag}
\begin{DoxyItemize}
\item \#output function\-: \-When an \-I\-P packet, after header compression, is too big to fit in a 802.\-15.\-4 frame, it is fragmented in several packets which are sent successively over the radio. \-The packets are formatted as defined in \-R\-F\-C 4944. \-Only the first fragment contains the \-I\-P/\-U\-D\-P compressed or uncompressed header fields.\end{DoxyItemize}
\begin{DoxyItemize}
\item \#input function\-: \-This function takes care of fragment reassembly. \-We do not assume that the fragments are received in order. \-When reassembly of a packet is ongoing, we discard any non fragmented packet or fragment from another packet. \-Reassembly times out after \#\-S\-I\-C\-S\-L\-O\-W\-P\-A\-N\-\_\-\-R\-E\-A\-S\-S\-\_\-\-M\-A\-X\-A\-G\-E = 20s.\end{DoxyItemize}
\begin{DoxyNote}{\-Note}
\-Fragmentation support is enabled by setting the \#\-S\-I\-C\-S\-L\-O\-W\-P\-A\-N\-\_\-\-C\-O\-N\-F\-\_\-\-F\-R\-A\-G compilation option.

\-As we do not support complex buffer allocation mechanism, for now we define a new 1280 bytes buffer (\#sicslowpan\-\_\-buf) to reassemble packets. \-At reception, once all the fragments are received, we copy the packet to \#uip\-\_\-buf, set \#uip\-\_\-len, and call \#tcpip\-\_\-input.

\#\-M\-A\-C\-\_\-\-M\-A\-X\-\_\-\-P\-A\-Y\-L\-O\-A\-D defines the maximum payload length in a 802.\-15.\-4 frame. \-For now it is constant and equal to 102 bytes (the 802.\-15.\-4 frame can be maximum 127 bytes long, and the header 25 bytes long).
\end{DoxyNote}
\hypertarget{a00056_hc}{}\paragraph{\-Header Compression}\label{a00056_hc}
{\bfseries \-Compression schemes}\par
 \-The \#\-S\-I\-C\-S\-L\-O\-W\-P\-A\-N\-\_\-\-C\-O\-N\-F\-\_\-\-C\-O\-M\-P\-R\-E\-S\-S\-I\-O\-N compilation option defines the compression scheme supported. \-We support \-H\-C1, \-H\-C01, and \-I\-Pv6 compression. \-H\-C1 and \-I\-Pv6 compression are defined in \-R\-F\-C4944, \-H\-C01 in draft-\/hui-\/6lowpan-\/hc. \-What we call \-I\-Pv6 compression means sending packets with no compression, and adding the \-I\-Pv6 dispatch before the \-I\-Pv6 header.\par
 \-If at compile time \-I\-Pv6 \char`\"{}compression\char`\"{} is chosen, packets sent will never be compressed, and compressed packets will not be processed at reception.\par
 \-If at compile time either \-H\-C1 or \-H\-C01 are chosen, we will try to compress all fields at sending, and will accept packets compressed with the chosen scheme, as well as uncompressed packets.\par
 \-Note that \-H\-C1 and \-H\-C01 supports are mutually exclusive. \-H\-C01 should soon deprecate \-H\-C1.

{\bfseries \-Compression related functions}\par
 \-When a packet is received, the \#input function is called. \-Fragmentation issues are handled, then we check the dispatch byte\-: if it is \-I\-Pv6, we treat the packet inline. \-If it is \-H\-C1 or \-H\-C01, the corresponding decompression function (\#uncompress\-\_\-hdr\-\_\-hc1 or \#uncompress\-\_\-hdr\-\_\-hc01) is called.\par
 \-When a packet needs to be sent, we try to compress it. \-If only the \-I\-Pv6 compression support is enabled, we just add the \-I\-Pv6 dispatch before the 802.\-15.\-4 payload. \-If \-H\-C1 or \-H\-C01 support is enabled, we call the corresponding compression function (\#compress\-\_\-hdr\-\_\-hc1 or \#compress\-\_\-hdr\-\_\-hc01) to compress the packet as much as possible.

{\bfseries \-H\-C1 comments}\par
 \-In \-H\-C1, if the \-I\-Pv6 flow label is not compressed, we would need to copy the fields after the flow label starting in the middle of a byte (the flow label is 20 bits long). \-To avoid this, we compress the packets only if all fields can be compressed. \-If we cannot, we use the \-I\-Pv6 dispatch and send all headers fields inline. \-This behavior is the one defined in draft-\/hui-\/6lowpan-\/interop-\/00.\par
 \-In the same way, if the packet is an \-U\-D\-P packet, we compress the \-U\-D\-P header only if all fields can be compressed.\par
 \-Note that \-H\-C1 can only compress unicast link local addresses. \-For this reason, we recommend using \-H\-C01.

{\bfseries \-H\-C01 comments}\par
 \-H\-C01 uses address contexts to enable compression of global unicast addresses. \-All nodes must share context (namely the global prefixes in use) to compress and uncompress such addresses successfully. \-The context number is defined by 2 bits. \-Context 00 is reserved for the link local context. \-Other contexts have to be distributed within the \-Lo\-W\-P\-A\-N dynamically, by means of \-N\-D extensions yet to be defined.\par
 \-Until then, if you want to test global address compression, you need to configure the global contexts manually.



 